\documentclass[12pt,letterpaper]{article}
\usepackage{graphicx,textcomp}
\usepackage{natbib}
\usepackage{setspace}
\usepackage{fullpage}
\usepackage{color}
\usepackage[reqno]{amsmath}
\usepackage{amsthm}
\usepackage{fancyvrb}
\usepackage{amssymb,enumerate}
\usepackage[all]{xy}
\usepackage{endnotes}
\usepackage{lscape}
\newtheorem{com}{Comment}
\usepackage{float}
\usepackage{hyperref}
\newtheorem{lem} {Lemma}
\newtheorem{prop}{Proposition}
\newtheorem{thm}{Theorem}
\newtheorem{defn}{Definition}
\newtheorem{cor}{Corollary}
\newtheorem{obs}{Observation}
\usepackage[compact]{titlesec}
\usepackage{dcolumn}
\usepackage{tikz}
\usetikzlibrary{arrows}
\usepackage{multirow}
\usepackage{xcolor}
\newcolumntype{.}{D{.}{.}{-1}}
\newcolumntype{d}[1]{D{.}{.}{#1}}
\definecolor{light-gray}{gray}{0.65}
\usepackage{url}
\usepackage{listings}
\usepackage{color}

\definecolor{codegreen}{rgb}{0,0.6,0}
\definecolor{codegray}{rgb}{0.5,0.5,0.5}
\definecolor{codepurple}{rgb}{0.58,0,0.82}
\definecolor{backcolour}{rgb}{0.95,0.95,0.92}

\lstdefinestyle{mystyle}{
	backgroundcolor=\color{backcolour},   
	commentstyle=\color{codegreen},
	keywordstyle=\color{magenta},
	numberstyle=\tiny\color{codegray},
	stringstyle=\color{codepurple},
	basicstyle=\footnotesize,
	breakatwhitespace=false,         
	breaklines=true,                 
	captionpos=b,                    
	keepspaces=true,                 
	numbers=left,                    
	numbersep=5pt,                  
	showspaces=false,                
	showstringspaces=false,
	showtabs=false,                  
	tabsize=2
}
\lstset{style=mystyle}
\newcommand{\Sref}[1]{Section~\ref{#1}}
\newtheorem{hyp}{Hypothesis}

\title{Problem Set 3}
\date{Due: March 24, 2024}
\author{Applied Stats II}


\begin{document}
	\maketitle
	\section*{Instructions}
	\begin{itemize}
	\item Please show your work! You may lose points by simply writing in the answer. If the problem requires you to execute commands in \texttt{R}, please include the code you used to get your answers. Please also include the \texttt{.R} file that contains your code. If you are not sure if work needs to be shown for a particular problem, please ask.
\item Your homework should be submitted electronically on GitHub in \texttt{.pdf} form.
\item This problem set is due before 23:59 on Sunday March 24, 2024. No late assignments will be accepted.
	\end{itemize}

	\vspace{.25cm}
\section*{Question 1}
\vspace{.25cm}
\noindent We are interested in how governments' management of public resources impacts economic prosperity. Our data come from \href{https://www.researchgate.net/profile/Adam_Przeworski/publication/240357392_Classifying_Political_Regimes/links/0deec532194849aefa000000/Classifying-Political-Regimes.pdf}{Alvarez, Cheibub, Limongi, and Przeworski (1996)} and is labelled \texttt{gdpChange.csv} on GitHub. The dataset covers 135 countries observed between 1950 or the year of independence or the first year forwhich data on economic growth are available ("entry year"), and 1990 or the last year for which data on economic growth are available ("exit year"). The unit of analysis is a particular country during a particular year, for a total $>$ 3,500 observations. 

\begin{itemize}
	\item
	Response variable: 
	\begin{itemize}
		\item \texttt{GDPWdiff}: Difference in GDP between year $t$ and $t-1$. Possible categories include: "positive", "negative", or "no change"
	\end{itemize}
	\item
	Explanatory variables: 
	\begin{itemize}
		\item
		\texttt{REG}: 1=Democracy; 0=Non-Democracy
		\item
		\texttt{OIL}: 1=if the average ratio of fuel exports to total exports in 1984-86 exceeded 50\%; 0= otherwise
	\end{itemize}
	
\end{itemize}
\newpage
\noindent Please answer the following questions:

\begin{enumerate}
	\item Construct and interpret an unordered multinomial logit with \texttt{GDPWdiff} as the output and "no change" as the reference category, including the estimated cutoff points and coefficients.
	
	\begin{lstlisting}[language=R]
		
		gdp_data$GDPWdiff_New <- factor(ifelse(gdp_data$GDPWdiff > 0, "positive", ifelse(gdp_data$GDPWdiff < 0, "negative", "no change")),
		levels = c("positive", "no change", "negative"),
		labels = c("positive", "no change", "negative"))
		
		gdp_data$REG <- factor(gdp_data$REG, levels = c(1, 0),
		labels = c("Democracy", "Non-Democracy"))
				
		gdp_data$OIL <- factor(gdp_data$OIL, levels = c(1, 0), labels = c("Exceed 50%", "Otherwise"))
		
		# Fit the multinomial logistic regression model on the  data
		
		gdp_data$REG <- relevel(gdp_data$REG, ref = "Non-Democracy")
		gdp_data$OIL <- relevel(gdp_data$OIL, ref = "Otherwise")
		gdp_data$GDPWdiff_New <- relevel(gdp_data$GDPWdiff_New , ref = "no change")
		
		multinom_model1 <- multinom(GDPWdiff_New ~ REG + OIL + COUNTRY, data = gdp_data)
		summary(multinom_model1)
		exp(coef(multinom_model1))
		
	\end{lstlisting}
	
The results are as follows:

		\begin{lstlisting}[language=R]
			
			> summary(multinom_model1)
			Call:
			multinom(formula = GDPWdiff_New ~ REG + OIL + COUNTRY, data = gdp_data)
			
			Coefficients:
			(Intercept) REGDemocracy OILExceed 50%    COUNTRY
			positive    3.015581    0.3550310      8.339428 0.03550780
			negative    2.900080    0.3539634      8.384195 0.02447895
			
			Std. Errors:
			(Intercept) REGDemocracy OILExceed 50%    COUNTRY
			positive   0.4045963    0.8686107    0.05903403 0.01064706
			negative   0.4056846    0.8702220    0.05903378 0.01066312
			
			Residual Deviance: 4568.055 
			AIC: 4584.055 
			> exp(coef(multinom_model1))
			(Intercept) REGDemocracy OILExceed 50%  COUNTRY
			positive    20.40094     1.426225      4185.695 1.036146
			negative    18.17560     1.424703      4377.332 1.024781
			
		\end{lstlisting}
		
	
	\item Construct and interpret an ordered multinomial logit with \texttt{GDPWdiff} as the outcome variable, including the estimated cutoff points and coefficients.
	
	\begin{lstlisting}[language=R]
		
		# ORDERED - PROPORTIONAL ODDS
		
		ordered_model <- polr(GDPWdiff_New ~ REG + OIL + COUNTRY, data = gdp_data, Hess=TRUE)
		
		# Print the summary of the model
		summary(ordered_model)
				
	\end{lstlisting}
	
	The results are as follows:
	
		\begin{lstlisting}[language=R]
		
Coefficients:
Value Std. Error  t value
REGDemocracy   0.004128   0.085548  0.04825
OILExceed 50%  0.077470   0.117196  0.66103
COUNTRY       -0.010252   0.001123 -9.13108

Intercepts:
Value    Std. Error t value 
no change|positive  -6.2318   0.2640   -23.6012
positive|negative    0.1769   0.0770     2.2972

Residual Deviance: 4606.003 
AIC: 4616.003 
		
	\end{lstlisting}
	

\end{enumerate}

\section*{Question 2} 
\vspace{.25cm}

\noindent Consider the data set \texttt{MexicoMuniData.csv}, which includes municipal-level information from Mexico. The outcome of interest is the number of times the winning PAN presidential candidate in 2006 (\texttt{PAN.visits.06}) visited a district leading up to the 2009 federal elections, which is a count. Our main predictor of interest is whether the district was highly contested, or whether it was not (the PAN or their opponents have electoral security) in the previous federal elections during 2000 (\texttt{competitive.district}), which is binary (1=close/swing district, 0="safe seat"). We also include \texttt{marginality.06} (a measure of poverty) and \texttt{PAN.governor.06} (a dummy for whether the state has a PAN-affiliated governor) as additional control variables. 

\begin{enumerate}
	\item [(a)]
	Run a Poisson regression because the outcome is a count variable. Is there evidence that PAN presidential candidates visit swing districts more? Provide a test statistic and p-value.
	
	\begin{lstlisting}[language=R]
	
	# load data
	mexico_elections <- read.csv("https://raw.githubusercontent.com/ASDS-TCD/StatsII_Spring2024/main/datasets/MexicoMuniData.csv")
	
	# Poisson regression
	mod.ps <- glm(PAN.visits.06 ~ competitive.district + marginality.06 + PAN.governor.06, data = mexico_elections, family = poisson)
	summary(mod.ps)
	
	# interpreting outputs
	cfs <- coef(mod.ps)
	cfs
	
	\end{lstlisting}

	The results are as follows:
	
	\begin{lstlisting}[language=R]
	
	
	Coefficients:
	Estimate Std. Error z value Pr(>|z|)    
	(Intercept)          -3.81023    0.22209 -17.156   <2e-16 ***
	competitive.district -0.08135    0.17069  -0.477   0.6336    
	marginality.06       -2.08014    0.11734 -17.728   <2e-16 ***
	PAN.governor.06      -0.31158    0.16673  -1.869   0.0617 .  
	---
	(Dispersion parameter for poisson family taken to be 1)
	
	Null deviance: 1473.87  on 2406  degrees of freedom
	Residual deviance:  991.25  on 2403  degrees of freedom
	AIC: 1299.2
	
	Number of Fisher Scoring iterations: 7
	
	> 
	> # interpreting outputs
	> cfs <- coef(mod.ps)
	> cfs
	(Intercept) competitive.district       marginality.06 
	-3.81023498          -0.08135181          -2.08014361 
	PAN.governor.06 
	-0.31157887 
	
	\end{lstlisting}
		
	\item [(b)]
	Interpret the \texttt{marginality.06} and \texttt{PAN.governor.06} coefficients.
	
\textbf{	Interpretation:}  

For a one unit change in the predictor \texttt{marginality.06} coefficient, the difference in the logs of expected counts for the number of times the winning PAN presidential candidate visits is expected to change by -2.08014, given the other predictor variables in the model are held constant.

For a one unit change in the predictor \texttt{PAN.governor.06} coefficient, the difference in the logs of expected counts for the number of times the winning PAN presidential candidate visits is expected to change by -0.31158, given the other predictor variables in the model are held constant.

The \texttt{marginality.06} coefficient appears to be statistically significant, whereas the \texttt{PAN.governor.06} coefficient does not appear to be so.

	\item [(c)]
	Provide the estimated mean number of visits from the winning PAN presidential candidate for a hypothetical district that was competitive (\texttt{competitive.district}=1), had an average poverty level (\texttt{marginality.06} = 0), and a PAN governor (\texttt{PAN.governor.06}=1).
	
	\begin{flalign*}
	&\lambda = \exp(\beta_0 + \beta_1 \times competitive.district + \beta_2 \times marginality.06 + \beta_3 \times PAN.governor.06) & \\
	&\lambda = \exp(\beta_0 + \beta_1 \times 1 + \beta_2 \times 0 + \beta_3 \times 1) & \\
	&\lambda = \exp(-3.81023 - 0.08135 \times 1 - 2.08014 \times 0 - 0.31158 \times 1) & \\
	&\lambda = \exp(-3.81023 - 0.08135 - 0.31158) & \\
	&\lambda = \exp(-4.20316) & \\
	&\text{Estimated Mean Visits} = \exp(-4.20316) & \\
	&\text{Estimated Mean Visits} \approx 0.015 &
\end{flalign*}

	
\end{enumerate}

\end{document}
